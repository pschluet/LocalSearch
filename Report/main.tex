\documentclass[sigconf]{acmart}

\settopmatter{printacmref=false}
\renewcommand\footnotetextcopyrightpermission[1]{}
\pagestyle{plain}

\usepackage[ruled, linesnumbered]{algorithm2e} % For algorithms
\usepackage{booktabs} % For formal tables
\newcommand{\bigo}[1]{\mathcal{O}(#1)}

\begin{document}

\title{Local Search Algorithms for Minimum Vertex Cover}

\author{Paul Schlueter}
\affiliation{%
  \institution{College of Computing\\Georgia Institute of Technology}
}
\email{paul@paulschlueter.com}

\begin{abstract}
The decision version of vertex cover is NP-Complete.
\end{abstract}

\keywords{CSE 6140, Minimum Vertex Cover, Local Search}

\maketitle
 
\section{Introduction} \label{sec:intro}
See \cite{cai2013numvc} for details.

\section{Problem Definition}

\section{Related Work}

\section{Algorithms}
Both of the local search algorithms implemented are similar in their top level structure, however, they differ in how they implement certain steps. Therefore, I will present the structure for the first algorithm then discuss how the second algorithm differs.
\subsection{LOCAL-SEARCH-VC-1}
The basic framework of this algorithm is based upon the decisional version of the vertex cover problem. It is an iterated $k$-vertex cover algorithm. We first determine an initial vertex cover of some integer size $k$. One vertex is then removed from this vertex cover, and two vertices are exchanged in and out until a new vertex cover is found of size $k - 1$.

The methods by which vertices are chosen to be removed and exchanged are based on a $score$ function which is based on a $cost$ function. The $cost$ function for a given graph, $G$, and candidate vertex cover solution, $C$, is as follows:
\begin{equation*}
	cost(G,C) = \text{the number of edges not covered by $C$}
\end{equation*}
where a lower cost is preferred.

The $score$ for a given vertex, $v$, is the increase in $cost$ that results from removing $v$ from the candidate vertex cover solution, $C$. It is defined as:
\begin{equation*}
	score(v) = cost(G,C) - cost(G,C')
\end{equation*}
where $C' = C \setminus \{v\}$. This $score$ is similar to the $dscore$ in REFERENCE HERE except that it is only calculated for vertices in $C$ to save computation time (as opposed to calculating the $dscore$ for all vertices in the graph).

Algorithm \ref{alg:ls1} shows the overall structure of the algorithm. The APPROX-VERTEX-COVER algorithm REFERENCE HERE used to find the initial vertex cover is a greedy approximation algorithm detailed in Algorithm \ref{alg:avc}.

\begin{algorithm}[h]
	\SetAlgoNoLine
	\KwIn{Graph with vertex set $V$ and edge set $E$}
	\KwOut{Set of vertices, $C*$, that form a vertex cover}
	$C$ = initial vertex cover from APPROX-VERTEX-COVER\\
	\While{elapsed time < cutoff time limit}{
		\If{$C$ is a vertex cover}{
			$C*$ = $C$\\
			Remove the vertex in $C$ with the lowest $score$\\
		}
		$u$ = the vertex in $C$ with the lowest $score$\\
		$C$ = $C$ $\setminus$ \{$u$\}\\
		$v$ = a random uncovered edge in $V \setminus C$\\
		$C$ = $C$ $\cup$ \{$v$\}
	}
	\Return $C*$
	\caption{LOCAL-SEARCH-VC-1}
	\label{alg:ls1}
\end{algorithm}

\begin{algorithm}[h]
	\SetAlgoNoLine
	\KwIn{Graph with vertex set $V$ and edge set $E$}
	\KwOut{Set of vertices, $C$, that form a vertex cover}
	$C$ = $\emptyset$\\
	$E'$ = $E$\\
	\While{$E' \neq \emptyset$}{
		let ($u$,$v$) be an arbitrary edge of $E'$\\
		$C$ = $C$ $\cup$ \{$u$,$v$\}\\
		remove from $E'$ every edge incident on either $u$ or $v$
	}
	\Return $C$
	
	\caption{APPROX-VERTEX-COVER}
	\label{alg:avc}
\end{algorithm}

Given a graph with vertex set $V$ and edge set $E$, the time complexity for a single time through the \textbf{while} loop in Algorithm \ref{alg:ls1} is $\bigo{V * E}$. The most expensive operation that drives this time complexity is the calculation of the scores for each of the vertices in the candidate vertex cover solution, $C$. To update the scores, we must loop through each of the vertices, $v_i$, in $C$, and inside this loop, we must loop through each of the edges connected to $v_i$. The space complexity for Algorithm \ref{alg:ls1} is $\bigo{V + E}$ because we must store the graph itself.

The time complexity for Algorithm \ref{alg:avc} is $\bigo{V + E}$ REFERENCE HERE, and its space complexity is $\bigo{V + E}$ because we must store the graph.  

\section{Empirical Evaluation}

\begin{table}[h]
	\caption{Algorithm Performance}
	\label{algperf}
	\begin{tabular}{lrrrrrr}
		\toprule
		& \multicolumn{3}{l}{Local Search 1} & \multicolumn{3}{l}{Local Search 2} \\ \midrule
		Dataset & Time(s)    & VC Size   & Rel Error  & Time (s)   & VC Size   & Rel Error  \\ \midrule
		as-22july06    & 29.49   & 3364      & 0.02   & 4.38   & 3373      & 0.02   \\
		delaunay n10    & 0.34   & 759      & 0.08   & 0.20   & 761      & 0.08   \\
		email    & 0.51   & 635      & 0.07   & 0.25   & 632      & 0.06   \\
		football    & 0.06   & 96      & 0.02   & 0.06   & 96      & 0.02   \\
		hep-th    & 9.05   & 4175      & 0.06   & 2.94   & 4155      & 0.06   \\
		jazz    & 0.10   & 162      & 0.02   & 0.10   & 162      & 0.02   \\
		karate    & 0.05   & 14.0      & 0.0   & 0.05   & 14.0      & 0.0   \\
		netscience    & 0.44   & 929      & 0.03   & 0.27   & 928      & 0.03   \\
		power    & 3.05   & 2370      & 0.08   & 0.79   & 2344      & 0.06   \\
		star    & 38.17   & 7033      & 0.02   & 9.18   & 7188      & 0.04   \\
		star2    & 31.30   & 4867      & 0.07   & 4.12   & 4867      & 0.07   \\
		\bottomrule
	\end{tabular}
\end{table}

\section{Discussion}

\section{Conclusion}
See Section \ref{sec:intro}.

\bibliographystyle{ACM-Reference-Format}
\bibliography{citations} 

\end{document}